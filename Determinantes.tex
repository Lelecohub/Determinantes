\documentclass[12pt]{article}
\usepackage{amsmath}
\usepackage{multicol}
\usepackage[latin1]{inputenc}

\title{F\'{o}rmula de Leibniz para Determinantes}
\author{Enri Lopes Iwasaki}
\date{22/04/2023}

\begin{document}
\maketitle

% EXERCÍCIO I
1) Deduza o determinande 4x4 usando a f\'{o}rmula:   

$$\operatorname{det}(A)=\sum_{\sigma \in S_{n}}  \left(\prod_{i=1}^{n} (-1)^ {\operatorname{sgn}(\sigma)}  ai \sigma (i) \right)$$

%PERMUTAÇÕES
\paragraph{}
\begin{center}
    Permuta\c{c}\~{o}es
\end{center}
$
\begin{aligned} 
    & S_{4}= {\{(1,2,3,4),(1,2,4,3),(1,3,2,4),(1,3,4,2),(1,4,2,3),(1,4,3,2),} \\ & \qquad \quad 
    (2,1,3,4),(2,1,4,3),(2,3,1,4),(2,3,4,1),(2,4,1,3),(2,4,3,1) \\ & \qquad \quad 
    (3,1,2,4),(3,1,4,2),(3,2,1,4),(3,2,4,1),(3,4,1,2),(3,4,2,1), \\ & \qquad \quad 
    (4,1,2,3),(4,1,3,2),(4,2,1,3),(4,2,3,1),(4,3,1,2),(4,3,2,1)\}
\end{aligned}
$

%SGN
\begin{center}
\begin{multicols}{2}
    $\operatorname{sgn}(1,2,3,4)=0 \\ \operatorname{sgn}(1,2,4,3)=1 \\
    \operatorname{sgn}(1,3,2,4)=1 \\ \operatorname{sgn}(1,3,4,2)=2 \\
    \operatorname{sgn}(1,4,2,3)=2 \\ \operatorname{sgn}(1,4,3,2)=3 \\ 
    \operatorname{sgn}(2,1,3,4)=1 \\ \operatorname{sgn}(2,1,4,3)=2 \\ 
    \operatorname{sgn}(2,3,1,4)=2 \\ \operatorname{sgn}(2,3,4,1)=3 \\ 
    \operatorname{sgn}(2,4,1,3)=3 \\ \operatorname{sgn}(2,4,3,1)=4 \\ 
    \operatorname{sgn}(3,1,2,4)=2 \\ \operatorname{sgn}(3,1,4,2)=3 \\ 
    \operatorname{sgn}(3,2,1,4)=3 \\ \operatorname{sgn}(3,2,4,1)=4 \\ 
    \operatorname{sgn}(3,4,1,2)=4 \\ \operatorname{sgn}(3,4,2,1)=5 \\ 
    \operatorname{sgn}(4,1,2,3)=3 \\ \operatorname{sgn}(4,1,3,2)=4 \\ 
    \operatorname{sgn}(4,2,1,3)=4 \\ \operatorname{sgn}(4,2,3,1)=5 \\ 
    \operatorname{sgn}(4,3,1,2)=5 \\ \operatorname{sgn}(4,3,2,1)=6$
\end{multicols}
\end{center}


% RESOLUÇÃO - PT I
\paragraph{}
    $$n = 4$$
    $$\operatorname{det}(A)=\sum_{\sigma \in S_{4}}  \left(\prod_{i=1}^{4} (-1)^ {\operatorname{sgn}(\sigma)}  ai \sigma (i) \right) = 
    \prod_{i=1}^{4} (-1)^ {\operatorname{sgn}'1, 2, 3, 4'}  ai_{'1, 2, 3, 4' (i)} + $$
    
    $$\prod_{i=1}^{4} (-1)^ {\operatorname{sgn}(1, 2, 4, 3)}  ai_{(1, 2, 4, 3) (i)} + 
    \prod_{i=1}^{4} (-1)^ {\operatorname{sgn}(1, 3, 2, 4)}  ai_{(1, 3, 2, 4) (i)} + 
    \prod_{i=1}^{4} (-1)^ {\operatorname{sgn}(1, 3, 4, 2)}  ai_{(1, 3, 4, 2) (i)} + $$
    
    $$\prod_{i=1}^{4} (-1)^ {\operatorname{sgn}(1, 4, 2, 3)}  ai_{(1, 4, 2, 3) (i)} + 
    \prod_{i=1}^{4} (-1)^ {\operatorname{sgn}(1, 4, 3, 2)}  ai_{(1, 4, 3, 2) (i)} + 
    \prod_{i=1}^{4} (-1)^ {\operatorname{sgn}(2, 1, 3, 4)}  ai_{(2, 1, 3, 4) (i)} + $$
    
    $$\prod_{i=1}^{4} (-1)^ {\operatorname{sgn}(2, 1, 4, 3)}  ai_{(2, 1, 4, 3) (i)} + 
    \prod_{i=1}^{4} (-1)^ {\operatorname{sgn}(2, 3, 1, 4)}  ai_{(2, 3, 1, 4) (i)} + 
    \prod_{i=1}^{4} (-1)^ {\operatorname{sgn}(2, 3, 4, 1)}  ai_{(2, 3, 4, 1) (i)} + $$
    
    $$\prod_{i=1}^{4} (-1)^ {\operatorname{sgn}(2, 4, 1, 3)}  ai_{(2, 4, 1, 3) (i)} + 
    \prod_{i=1}^{4} (-1)^ {\operatorname{sgn}(2, 4, 3, 1)}  ai_{(2, 4, 3, 1) (i)} + 
    \prod_{i=1}^{4} (-1)^ {\operatorname{sgn}(3, 1, 2, 4)}  ai_{(3, 1, 2, 4) (i)} + $$
    
    $$\prod_{i=1}^{4} (-1)^ {\operatorname{sgn}(3, 1, 4, 2)}  ai_{(3, 1, 4, 2) (i)} + 
    \prod_{i=1}^{4} (-1)^ {\operatorname{sgn}(3, 2, 1, 4)}  ai_{(3, 2, 1, 4) (i)} + 
    \prod_{i=1}^{4} (-1)^ {\operatorname{sgn}(3, 2, 4, 1)}  ai_{(3, 2, 4, 1) (i)} + $$
    
    $$\prod_{i=1}^{4} (-1)^ {\operatorname{sgn}(3, 4, 1, 2)}  ai_{(3, 4, 1, 2) (i)} + 
    \prod_{i=1}^{4} (-1)^ {\operatorname{sgn}(3, 4, 2, 1)}  ai_{(3, 4, 2, 1) (i)} + 
    \prod_{i=1}^{4} (-1)^ {\operatorname{sgn}(4, 1, 2, 3)}  ai_{(4, 1, 2, 3) (i)} + $$
    
    $$\prod_{i=1}^{4} (-1)^ {\operatorname{sgn}(4, 1, 3, 2)}  ai_{(4, 1, 3, 2) (i)} + 
    \prod_{i=1}^{4} (-1)^ {\operatorname{sgn}(4, 2, 1, 3)}  ai_{(4, 2, 1, 3) (i)} + 
    \prod_{i=1}^{4} (-1)^ {\operatorname{sgn}(4, 2, 3, 1)}  ai_{(4, 2, 3, 1) (i)} + $$
    
    $$\prod_{i=1}^{4} (-1)^ {\operatorname{sgn}(4, 3, 1, 2)}  ai_{(4, 3, 1, 2) (i)} + 
    \prod_{i=1}^{4} (-1)^ {\operatorname{sgn}(4, 3, 2, 1)}  ai_{(4, 3, 2, 1) (i)} $$

\newpage


%POSIÇÕES
\paragraph{}
\begin{center}
    Posi\c{c}\~{o}es
\\

\begin{multicols}{3}
    $'1,2,3,4'(1)= 1 \\ '1,2,3,4'(2)= 2 \\ '1,2,3,4'(3)= 3 \\ '1,2,3,4'(4)= 4 \\ '1,2,4,3'(1)= 1 \\
    '1,2,4,3'(2)= 2 \\ '1,2,4,3'(3)= 4 \\ '1,2,4,3'(4)= 3 \\ '1,3,2,4'(1)= 1 \\ '1,3,2,4'(2)= 3 \\
    '1,3,2,4'(3)= 2 \\ '1,3,2,4'(4)= 4 \\ '1,3,4,2'(1)= 1 \\ '1,3,4,2'(2)= 3 \\ '1,3,4,2'(3)= 4 \\ 
    '1,3,4,2'(4)= 2 \\ '1,4,2,3'(1)= 1 \\ '1,4,2,3'(2)= 4 \\ '1,4,2,3'(3)= 2 \\ '1,4,2,3'(4)= 3 \\ 
    '2,3,1,4'(1)= 2 \\ '2,3,1,4'(2)= 3 \\ '2,3,1,4'(3)= 1 \\ '2,3,1,4'(4)= 4 \\ '2,3,4,1'(1)= 2 \\
    '2,3,4,1'(2)= 3 \\ '2,3,4,1'(3)= 4 \\ '2,3,4,1'(4)= 1 \\ '2,4,1,3'(1)= 2 \\ '2,4,1,3'(2)= 4 \\
    '2,4,1,3'(3)= 1 \\ '2,4,1,3'(4)= 3 \\ '2,4,3,1'(1)= 2 \\ '2,4,3,1'(2)= 4 \\ '2,4,3,1'(3)= 3 \\
    '2,4,3,1'(4)= 1 \\ '3,1,2,4'(1)= 3 \\ '3,1,2,4'(2)= 1 \\ '3,1,2,4'(3)= 2 \\ '3,1,2,4'(4)= 4 \\
    '3,1,4,2'(1)= 3 \\ '3,1,4,2'(2)= 1 \\ '3,1,4,2'(3)= 4 \\ '3,1,4,2'(4)= 2 \\ '3,2,1,4'(1)= 3 \\
    '3,2,1,4'(2)= 2 \\ '3,2,1,4'(3)= 1 \\ '3,2,1,4'(4)= 4 \\ '3,2,4,1'(1)= 3 \\ '3,2,4,1'(2)= 2 \\
    '3,2,4,1'(3)= 4 \\ '3,2,4,1'(4)= 1 \\ '3,4,1,2'(1)= 3 \\ '3,4,1,2'(2)= 4 \\ '3,4,1,2'(3)= 1 \\
    '3,4,1,2'(4)= 2 \\ '3,4,2,1'(1)= 3 \\ '3,4,2,1'(2)= 4 \\ '3,4,2,1'(3)= 2 \\ '3,4,2,1'(4)= 1 \\
    '4,1,2,3'(1)= 4 \\ '4,1,2,3'(2)= 1 \\ '4,1,2,3'(3)= 2 \\ '4,1,2,3'(4)= 3 \\ '4,1,3,2'(1)= 4 \\
    '4,1,3,2'(2)= 1 \\ '4,1,3,2'(3)= 3 \\ '4,1,3,2'(4)= 2 \\ '4,2,1,3'(1)= 4 \\ '4,2,1,3'(2)= 2 \\
    '4,2,1,3'(3)= 1 \\ '4,2,1,3'(4)= 3 \\ '4,2,3,1'(1)= 4 \\ '4,2,3,1'(2)= 2 \\ '4,2,3,1'(3)= 3 \\
    '4,2,3,1'(4)= 1 \\ '4,3,1,2'(1)= 4 \\ '4,3,1,2'(2)= 3 \\ '4,3,1,2'(3)= 1 \\ '4,3,1,2'(4)= 2 \\
    '4,3,2,1'(1)= 4 \\ '4,3,2,1'(2)= 3 \\ '4,3,2,1'(3)= 2 \\ '4,3,2,1'(4)= 1 $
\end{multicols}
\end{center}

\newpage

%RESOLUÇÃO - PT II
\paragraph{}
    $$\operatorname{det}(A)= \prod_{i=1}^{4} (-1)^0  ai_{(1, 2, 3, 4) (i)} + 
    \prod_{i=1}^{4} (-1)^1  ai_{(1, 2, 4, 3) (i)} +
    \prod_{i=1}^{4} (-1)^1  ai_{(1, 3, 2, 4) (i)} +$$
    
    $$\prod_{i=1}^{4} (-1)^2  ai_{ (1, 3, 4, 2) (i)} + 
    \prod_{i=1}^{4} (-1)^2  ai_{ (1, 4, 2, 3) (i)} +
    \prod_{i=1}^{4} (-1)^3  ai_{ (1, 4, 3, 2) (i)} +$$
    
    $$\prod_{i=1}^{4} (-1)^1  ai_{(2, 1, 3, 4) (i)} + 
    \prod_{i=1}^{4} (-1)^2  ai_{(2, 1, 4, 3) (i)} +
    \prod_{i=1}^{4} (-1)^2  ai_{(2, 3, 1, 4) (i)} +$$
    
    $$\prod_{i=1}^{4} (-1)^3  ai_{(2, 3, 4, 1) (i)} + 
    \prod_{i=1}^{4} (-1)^3  ai_{(2, 4, 1, 3) (i)} +
    \prod_{i=1}^{4} (-1)^4  ai_{(2, 4, 3, 1) (i)} +$$
    
    $$\prod_{i=1}^{4} (-1)^2  ai_{(3, 1, 2, 4) (i)} + 
    \prod_{i=1}^{4} (-1)^3  ai_{(3, 1, 4, 2) (i)} +
    \prod_{i=1}^{4} (-1)^3  ai_{(3, 2, 1, 4) (i)} +$$
    
    $$\prod_{i=1}^{4} (-1)^4  ai_{(3, 2, 4, 1) (i)} + 
    \prod_{i=1}^{4} (-1)^4  ai_{(3, 4, 1, 2) (i)} +
    \prod_{i=1}^{4} (-1)^5  ai_{(3, 4, 2, 1)(i)} +$$
    
    $$\prod_{i=1}^{4} (-1)^3  ai_{(4, 1, 2, 3) (i)} + 
    \prod_{i=1}^{4} (-1)^4  ai_{(4, 1, 3, 2) (i)} +
    \prod_{i=1}^{4} (-1)^4  ai_{(4, 2, 1, 3) (i)} +$$
    
    $$\prod_{i=1}^{4} (-1)^5  ai_{(4, 2, 3, 1) (i)} + 
    \prod_{i=1}^{4} (-1)^5  ai_{(4, 3, 1, 2) (i)} +
    \prod_{i=1}^{4} (-1)^6  ai_{(4, 3, 2, 1) (i)} +$$

\newpage


% RESULTADO
\paragraph{}
\[
\begin{aligned}
    \operatorname{det}(A) & =
    a_{11} a_{22} a_{33} a_{44} -a_{11} a_{22} a_{43} a_{34} -a_{11} a_{32} a_{23} 
    a_{44} +a_{11} a_{32} a_{43} a_{24} \\
    & +a_{11} a_{42} a_{23} a_{34} -a_{11} a_{42} a_{33} a_{24} -a_{21} a_{12} a_{33} a_{44} 
    +a_{21} a_{12} a_{43} a_{34} \\
    & +a_{21} a_{32} a_{13} a_{44} -a_{21} a_{32} a_{43} a_{14} -a_{21} a_{42} a_{13} a_{34} 
    +a_{21} a_{42} a_{33} a_{14} \\
    & +a_{31} a_{12} a_{23} a_{44} -a_{31} a_{12} a_{43} a_{24} -a_{31} a_{22} a_{13} a_{44} 
    +a_{31} a_{22} a_{43} a_{14} \\
    & +a_{31} a_{42} a_{13} a_{24} -a_{31} a_{42} a_{23} a_{14} -a_{41} a_{12} a_{23} a_{34} 
    +a_{41} a_{12} a_{33} a_{24} \\
    &  +a_{41} a_{22} a_{13} a_{34} -a_{41} a_{22} a_{33} a_{14} -a_{41} a_{32} a_{13} a_{24} 
    +a_{41} a_{32} a_{23} a_{14} 
\end{aligned}
\]
\begin{center}
    $$$$
    Resultado Final \\
\end{center}

\begin{center}
\fbox{
\begin{aligned}
    \operatorname{det}(A) & =a_{11} a_{22} a_{33} a_{44}+a_{11} a_{32} a_{43} a_{24}+a_{11} a_{42} a_{23} a_{34} +a_{21} a_{12} a_{43} a_{34}\\
    & +a_{21} a_{32} a_{13} a_{44}+a_{21} a_{42} a_{33} a_{14} +a_{31} a_{12} a_{23} a_{44}+a_{31} a_{22} a_{43} a_{14} \\
    & +a_{31} a_{42} a_{13} a_{24} +a_{41} a_{12} a_{33} a_{24}+a_{41} a_{22} a_{13} a_{34}+a_{41} a_{32} a_{23} a_{14} \\
    & -a_{11} a_{22} a_{43} a_{34}-a_{11} a_{32} a_{23} a_{44}-a_{11} a_{42} a_{33} a_{24} -a_{21} a_{12} a_{33} a_{44} \\
    & -a_{21} a_{32} a_{43} a_{14}-a_{21} a_{42} a_{13} a_{34} -a_{31} a_{12} a_{43} a_{24}-a_{31} a_{22} a_{13} a_{44}\\
    & -a_{31} a_{42} a_{23} a_{14} -a_{41} a_{12} a_{23} a_{34}-a_{41} a_{22} a_{33} a_{14}-a_{41} a_{32} a_{13} a_{24}\\
\end{aligned}
}
\end{center}
\newpage


%Exercicio 2
2) Calcule o determinante usando o que foi deduzido, de duas matrizes definidas pelo autor (det = 0 / det $\neq$ 0):

% a) Matriz 1: det(A) = 0
\paragraph{}
\begin{itemize}
    \item det = 0
\end{itemize}

\[
  A =
  \left[ {\begin{array}{cccc}
    1 & 1 & 1 & 1\\
    1 & 1 & 1 & 1\\
    1 & 1 & 1 & 1\\
    1 & 1 & 1 & 1
  \end{array} } \right]
\]

% RESOLUÇÃO MATRIZ 1
\[
\begin{aligned}
    \operatorname{det}(A) & = 1 \cdot 1 \cdot 1 \cdot 1 + 1 \cdot 1 \cdot 1 \cdot 1 + 1 \cdot 1 \cdot 1 
    \cdot 1 + 1 \cdot 1 \cdot 1 \cdot 1 + 1 \cdot 1 \cdot 1 \cdot 1 + 1 \cdot 1 \cdot 1 \cdot 1\\
    & + 1 \cdot 1 \cdot 1 \cdot 1 + 1 \cdot 1 \cdot 1 \cdot 1 + 1 \cdot 1 \cdot 1 \cdot 1 + 1 \cdot 1 
    \cdot 1 \cdot 1 + 1 \cdot 1 \cdot 1 \cdot 1 + 1 \cdot 1 \cdot 1 \cdot 1\\
    & - 1 \cdot 1 \cdot 1 \cdot 1 - 1 \cdot 1 \cdot 1 \cdot 1 - 1 \cdot 1 \cdot 1 \cdot 1 - 1 \cdot 1 
    \cdot 1 \cdot 1 - 1 \cdot 1 \cdot 1 \cdot 1 - 1 \cdot 1 \cdot 1 \cdot 1\\
    & - 1 \cdot 1 \cdot 1 \cdot 1 - 1 \cdot 1 \cdot 1 \cdot 1 - 1 \cdot 1 \cdot 1 \cdot 1 - 1 \cdot 1 
    \cdot 1 \cdot 1 - 1 \cdot 1 \cdot 1 \cdot 1 - 1 \cdot 1 \cdot 1 \cdot 1 \\ 
    \operatorname{det}(A) & = 1 + 1 + 1 + 1 + 1 + 1 + 1 + 1 + 1 + 1 + 1 + 1 - 1 - 1 - 1 - 1 - 1 - 1  \\ 
    & \quad \ - 1  - 1 - 1 - 1 - 1 - 1 \\
    \operatorname{det}(A) & = 12-12 \\
    \operatorname{det}(A) & = 0
\end{aligned}
\]


% b) Matriz 2: det(A) != 0
\paragraph{}
\begin{itemize}
    \item det $\neq$ 0
\end{itemize}
\[
  B =
  \left[ {\begin{array}{cccc}
    2 & 0 & 1 & 0\\
    0 & 2 & 0 & 1\\
    1 & 0 & 2 & 0\\
    0 & 1 & 0 & 2
  \end{array} } \right]
\]

\[
% RESOLUÇÃO MATRIZ 2
\begin{aligned}
    \operatorname{det}(B) & = 2 \cdot 2 \cdot 2 \cdot 2 + 2 \cdot 0 \cdot 0 \cdot 1 + 2 \cdot 1 \cdot 0 
    \cdot 0 + 0 \cdot 0 \cdot 0 \cdot 2 + 0 \cdot 0 \cdot 1 \cdot 2 + 0 \cdot 1 \cdot 2 \cdot 0 \\
    &  + 1 \cdot 0 \cdot 0 \cdot 2 + 1 \cdot 2 \cdot 0 \cdot 0 + 1 \cdot 1 \cdot 1 \cdot 1 + 0 \cdot 0 
    \cdot 2 \cdot 1 + 0 \cdot 2 \cdot 1 \cdot 0 + 0 \cdot 0 \cdot 0 \cdot 1 \\
    & - 0 \cdot 2 \cdot 0 \cdot 0 - 0 \cdot 0 \cdot 0 \cdot 2 - 2 \cdot 1 \cdot 2 \cdot 1 - 0 \cdot 0 
    \cdot 2 \cdot 2 - 1 \cdot 0 \cdot 0 \cdot 0 - 0 \cdot 1 \cdot 1 \cdot 0 \\
    & - 1 \cdot 0 \cdot 0 \cdot 0 - 1 \cdot 2 \cdot 1 \cdot 2 - 1 \cdot 1 \cdot 0 \cdot 0 - 0 \cdot 0 
    \cdot 0 \cdot 0 - 0 \cdot 2 \cdot 2 \cdot 0 - 0 \cdot 0 \cdot 1 \cdot 1\\ 
    \operatorname{det}(B) & = 16 + 0 + 0 + 0 + 0 + 0 + 0 + 0 + 1 + 0 + 0 + 0 - 0 - 0 - 4 - 0 - 0 - 0 \\ 
    & \quad \ - 0 - 4 - 0 - 0 - 0 - 0 \\
    \operatorname{det}(B) & = 17-8 \\
    \operatorname{det}(B) & = 9
\end{aligned}
\]

\end{document}
